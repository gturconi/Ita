% --------------------------------------------------------------
% This is all preamble stuff that you don't have to worry about.
% Head down to where it says "Start here"
% --------------------------------------------------------------
 
\documentclass[12pt,spanish]{article}
 
\usepackage[margin=1in]{geometry} 
\usepackage{amsmath,amsthm,amssymb}
\usepackage{mathtools}
\usepackage{enumitem}

% Para eliminar la numeración de secciones
\setcounter{secnumdepth}{0}
\frenchspacing

% Solución temporal al problema de los tildes, funciona bien sin instalar nada, pero habría que saber instalar polyglossia.
\usepackage[T1]{fontenc}
\usepackage{selinput}
\SelectInputMappings{%
  aacute={á},
  eacute={é},
  iacute={í},
  oacute={ó},
  uacute={ú},
  ntilde={ñ}
}

% Para simbolos de conjuntos usualmente usados. 
\newcommand{\N}{\mathbb{N}}
\newcommand{\Z}{\mathbb{Z}}
\newcommand{\R}{\mathbb{R}}
\newcommand{\K}{\mathbb{K}}
 
% Para estructurar el documento en teoremas, lemas, ejercicios, reflexiones, proposiciones, corolarios, y demostraciones.
\newenvironment{teorema}[2][Teorema]{\begin{trivlist}
\item[\hskip \labelsep {\bfseries #1}\hskip \labelsep {\bfseries #2.}]}{\end{trivlist}}
\newenvironment{lema}[2][Lema]{\begin{trivlist}
\item[\hskip \labelsep {\bfseries #1}\hskip \labelsep {\bfseries #2.}]}{\end{trivlist}}
\newenvironment{ejercicio}[2][Ejercicio]{\begin{trivlist}
\item[\hskip \labelsep {\bfseries #1}\hskip \labelsep {\bfseries #2.}]}{\end{trivlist}}
\newenvironment{reflexion}[2][Reflexión]{\begin{trivlist}
\item[\hskip \labelsep {\bfseries #1}\hskip \labelsep {\bfseries #2.}]}{\end{trivlist}}
\newenvironment{proposicion}[2][Proposición]{\begin{trivlist}
\item[\hskip \labelsep {\bfseries #1}\hskip \labelsep {\bfseries #2.}]}{\end{trivlist}}
\newenvironment{corolario}[2][Corolario]{\begin{trivlist}
\item[\hskip \labelsep {\bfseries #1}\hskip \labelsep {\bfseries #2.}]}{\end{trivlist}}
\newenvironment{demostracion}{{\emph{Demostración.}}}{\hfill $\blacksquare$ \\} 

% Para poder escribir \iff en math mode fácil.
\DeclareRobustCommand\iff{\;\Leftrightarrow\;}
 
\begin{document}
 
% --------------------------------------------------------------
%                         Start here
% --------------------------------------------------------------
 
\title{TP Final Arquitectura - Pilot}
\author{Gonzalo Turconi, Ignacio Cattoni, Redigonda Maximiliano}
\date{Enero 8, 2018}
 
\maketitle

\section{Introducción}
El objetivo de este projecto es generar un compilador para el lenguaje Pilot a lenguaje ensamblador ARM. \textbf{Pilot} es un lenguaje sencillo, del cual nosotros proveemos una descripción basada en las consignas del trabajo práctico.

Todas las sentencias en Pilot son de una linea. Cada linea comienza con una letra que indica la operación a ejecutarse, o con el nombre de una variable para una operación de asignación.

Las variables son cadenas alfanuméricas que comienzan con una letra en minúscula, y son seguidas obligatoriamente por un entero de etiqueta\footnote{Un entero de etiqueta es un número entero no negativo representable por un entero con signo de 32 bits. Cualquier entero entre $0$ y $2^{31}-1$ inclusive, sin ceros a izquierda.}. De esta manera:
\begin{itemize}
\item{a0, z24, x231, p2147483647, son nombres válidos, mientras que}
\item{a-1, b, romeosantos, 23z, w2147483648, no son nombres válidos}
\end{itemize}

\noindent Las operaciones soportadas son las siguientes:
\begin{itemize}
\item{Asignación: se caracteriza por comenzar con el nombre de una variable y seguir con una expresión. Su efecto es asignar el valor de la expresión en la variable.}
\item{Entrada: comienza con un caracter R seguido de el nombre de una variable. Su efecto es leer un entero por la entrada estándar y asignárselo a la variable.}
\item{Salida: comienza con un caracter O seguido de una variable o una expresión. Su efecto es imprimir el valor de la variable o de la expresión por pantalla.}
\item{Terminar: consiste de una única letra E, su efecto es el de terminar el programa.}
\item{Salto: se representa por una G inicial, seguida de un entero de etiqueta. Su efecto es el de interrumpir la ejecución del programa para resumirla en la etiqueta caracterizada por el entero.}
\item{Etiqueta: comienza con una L, y es seguida por un entero de etiqueta. Su efecto es definir la etiqueta caracterizada por el entero. Sólo puede haber una definición de la etiqueta por cada entero.}
\item{Salto condicional: se representa por una letra I, seguida de una expresión o variable, y termina en un entero de etiqueta. Su efecto es el de saltar a la etiqueta correspondiente si el valor de la expresión o variable es distinto de cero.}
\end{itemize}

\noindent Es importante notar que una expresión, está definida como una variable, una constante o un operador seguido de dos variables o constantes.

\section{Pasos a seguir}
Para realizar este projecto, necesitaremos:
\begin{enumerate}
\item{Familiarizarnos con el lenguaje Pilot.}
\item{Diagramar una posible implementación de alto nivel del compilador.}
\item{Crear un sistema de procesado linea por linea.}
\item{Interpretar cada linea y discriminarla por tipo de operación.}
\item{Obtener el conjunto de instrucciones equivalentes a cada sentencia de Pilot.}
\item{Mapear las sentencias en Pilot a cada  conjunto de instrucciones de ARM.}
\item{Hacer pruebas de correctitud creando diversos programas en Pilot y verificando que actuen segun lo esperado.}
\end{enumerate}

% --------------------------------------------------------------
%     You don't have to mess with anything below this line.
% --------------------------------------------------------------
 
\end{document}